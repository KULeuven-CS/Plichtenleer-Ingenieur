\documentclass{article}
\usepackage[utf8]{inputenc}

\usepackage[a4paper,
    top=3cm,
    bottom=3cm,
    outer=5cm,
    inner=5cm,
    heightrounded,
    marginparwidth=3cm,
    marginparsep=1cm]{geometry}

\usepackage[dutch]{babel}
\usepackage[colorlinks]{hyperref}
\usepackage[backend=bibtex,style=ieee]{biblatex}
\usepackage[hyperref]{ntheorem}
\usepackage{microtype}
\usepackage{amsmath,amsfonts,amssymb}
\usepackage{parskip}

\bibliography{main}

% Margin notes
\usepackage{marginnote}
\usepackage{mparhack}
\usepackage{marginfix}

\newcommand{\annotation}[1]{%
    \marginpar{\small\textit{#1}}}

% Parskip
\setlength{\parindent}{0pt}

\makeatletter
\def\@seccntformat#1{\llap{\csname the#1\endcsname\quad}}
\makeatother

% Title Page
\title{Plichtenleer van de ingenieur}
\date{\today}
\author{Dieter~Castel}

% Q&A
\theoremstyle{nonumberplain}
\newtheorem{question}{Vraag}

\theorembodyfont{}
\newtheorem{answer}{Antwoord}

\begin{document}
\maketitle
\tableofcontents
\newpage

\section{Inleiding}
De oplossingen van de vragen komen, tenzij waar anders vermeld, uit het boek dat wordt gebruikt voor dit vak: \textit{Business en ethiek, Spelregels voor ethisch ondernemen} \cite{gerwen2002business}. Paginaverwijzingen worden sterk aangeraden om het controleren of uitbreiden van een antwoord te vereenvoudigen.

\section{Vragen en antwoorden}

\begin{question}
(1)	Waarom zijn ethische codes zo belangrijk voor de professionele wereld van de ingenieurs? Bij welke vorm van ethisch argumenteren sluiten deze codes aan? 
\end{question}
\begin{answer}[Dieter]
	Ten eerste kan een ethische code sommige wetten expliciteren waardoor, bij twijfel, de code eenvoudig als handleiding gebruikt kan worden indien er zich betwistingen voordoen.
	Een uitgebreidere code kan echter ook meer informele handelswijzen omvatten.
	Daarmee kan de code ook eenvoudig gebruikt worden om te informeren over de waarden en normen die in het bedrijf gehanteerd worden, zowel intern als extern (naar de samenleving toe) .
	Verder kunnen de codes ook een strategisch voordeel opleveren door een ethisch label dat wordt verworven door het strikt opvolgen van de code.

	Ethische codes kunnen we onderbrengen onder het deontologisch argumenteren aangezien ze meestal de plichten en rechten van de betrokkenen beschrijven.
	\textit{(Hfdst.2 Bedrijfsethiek als ethiek van de onderneming, p.54-55, p.24)}
\end{answer}

\begin{question}
(2)	Waarom is de titel 'ingenieursethiek' beter geschikt dan 'deontologie van de ingenieur'?
\end{question}
\begin{answer}[Dieter]
	Deontologie is maar \'e\'en peiler van de ethiek. Zelfs als een ingenieur deontologisch correct handelt (Een contract volgen omdat dat zijn plicht is) kan hij of zij nog steeds onethisch handelen (Als dat contract onherstelbare schade berokkend aan de mens/natuur).
	\textit{(Hfdst. 1: Modellen van bedrijfsethiek en ethische argumenteren, p.22)}
\end{answer}

\begin{question}
(3)	Bespreek de ethische code van de KVIV .
\end{question}
\begin{answer}[Dieter]
	\begin{itemize}
		\item Technologie \& Maatschapij: Oordeelkundige inzet van technologie helpt de maatschapij.
		\item Verantwoordelijkheid als individu, groep, ingenieur, lid van maatschapij. 
		\item Integriteit in competenties, verhoudingen (loyaal), belangenconflicten, vertrouwlijkheid v. informatie, t.o.v. klanten
		\item Openheid: aanvaard eerlijke kritiek, open voor bijdrage andere beroepen, erkennen/verbeteren van eigen fouten
		\item Redlijkheid: Voorzichtig met invloed, respect voor geestelijke en lichamelijke integriteit van alle stakeholders.
		\item Burgerzin 
			\begin{itemize}
				\item Veel technologi\"en groeien uit de gehele maatschapij
				\item Volledige informatie (risico's, op gepaste ogenblik eigen mening)
				\item Bekomernis om duurzaamheid (van realisatie als milieu)
				\item Pro-actief voorstellen om kwaliteit/regelgeving te verbeteren
				\item Eerbiedigen verschillende culturen (integreren)
			\end{itemize} 
		\item Collegialiteit (Onderteunen elkaar, wedijveren met respect, erkenning, persoonlijke verantwoordelijkheid t.o.v. imago v. beroepsgroep)
	\end{itemize}

	\textit{(Uit het ingescande: `Ethische Leidraad voor de Ingenieur (KVIV)'.)}
\end{answer}

\begin{question}
(4)	Wat hebben whistleblowing en deontologische ethiek met elkaar te maken?
\end{question}
\begin{answer}[Dieter]
	In het geval van whistleblowing conflicteren twee normen: zwijgplicht en spreekplicht.
	Immers onder normale omstandigheden is het niet verantwoord om bedrijfsgeheimen naar buiten te brengen.
	Als er echter rechten van het publiek/de consument dreigen geschonden te worden door het bedrijf leid dit tot een conflict.
	Er moet echter eerst interne middelen worden aangewend om dit probleem op te lossen. Als dat geen resultaat opleveren moeten er nog de volgende overwegingen gemaakt worden. Zal de schade die berokkend wordt niet toevallig (of van voorbijgaande aard), aan het algemeen welzijn (of een onwetende partij) zijn en beschikt de werknemer over meer dan enkel vermoedens. Als dat allemaal het geval is kan whistleblowing legitiem zijn.
\end{answer}

\begin{question}
(5)	Geef een kritiek op een eenzijdig utilitaristische benadering van bedrijfsethiek.
\end{question}
\begin{answer}[Dieter]
	Ten eerste is het nut bepalen voor ALLE betrokken partijen soms praktisch moeilijk. Daarom moet er rekening worden gehouden met onzekerheid (bvb. met probabilisme via kansberekening).
	Ten tweede alternatieven vergelijken moet gebeuren door deze te kwantificeren (kosten-baten analyse). Sommige dingen zijn z\'e\'er moelijk kwantificeerbaar: mensenleven, uitsterven diersoort, \ldots
	\textit{(Hfdst. 1: Modellen van bedrijfethiek en ethisch argumenteren, p. 20)}
\end{answer}

\begin{question}
(6)	Wat kan men leren uit de Ford Pinto case in verband met ethisch argumenteren?
\end{question}
\begin{answer}[Dieter]
	Dat louter utilitaristisch denken duidelijk zijn beperkingen heeft.
	Het leven van een mens kwantificeren is uitermate moeilijk.
\end{answer}

\begin{question}
(7)	Wat is het verschil tussen deontologische ethiek en deugdenethiek?
\end{question}
\begin{answer}[Dieter]
	Deugden zijn inherent aan bepaalde praktijken.
	Ze moeten niet beoefend worden omdat het een plicht is (zoals in deontologische ethiek) maar omdat leden van een instituut door hun omgeving gestimuleerd worden tot het uitmuntend beoefenen van een gemeenschappelijke praktijk.
	In het beoefenen van die praktijk vinden ze voldoening en dienen het algemeen belang.
	Bovendien kan de deugdenethiek ook de rolintegriteit (rol van ingenieur) overschrijden.
	Als iemand integrale integriteit bereikt die verschillende verantwoordelijkheden in een coherent leven samenbrengt.
	\textit{(Hfdst. 1: Modellen van bedrijfsethiek en ethische argumenteren, p.32-35)}
\end{answer}

\begin{question}
(8)	Kan een bedrijf als bedrijf verantwoordelijk worden gesteld? Kan men een bedrijf strafrechtelijke vervolgen?
\end{question}
\begin{answer}[Dieter]
	Ja, indien het misdrijf het gevolg is van slechte \emph{standard operating procedures} en niet louter te wijten is aan een individuele fout.
	Bijvoorbeeld structureel slechte veiligheidsprocedures, onvoldoende kwaliteitscontrole, nalatigheid in onderhoud, \ldots
	Hoewel de casus met de Herald of Free Enterprise \textit{(p40-41)} een typisch voorbeeld zou kunnen zijn van structurele problemen werd P\&O toen niet vervolgd door de zeer recente overname van Towsend Thorsen.
	\textit{(Hfdst. 2 Bedrijfsethiek als ethiek van de onderneming, p.49-50)}
\end{answer}

\begin{question}
(9)	Toon aan waarom de theorie van de verplichte zorg de meest adequate benadering is van het probleem van de productverantwoordelijkheid.
\end{question}
\begin{answer}[Dieter]
	De theorie is vanuit de 3 grote ethische modellen te funderen:
	Ten eerste gaat de theorie van de verplichte zorg uit van het belang van de kwetsbare consument. 
	Volgens deze theorie heeft de producent de plicht om alle redelijke voorzorgen te nemen om de consument geen schade te berokkenen en daarmee volgt het een belangrijk deontologisch principe.
	Een persoon heeft namelijk het recht om beschermd te worden en is een doel op zich.
	Ook utilistisch is het principe te funderen aangezien iedereen, meestal, welvaart bij het aanvaarden van dit principe.
	Ten laatste is ook vanuit het rechtvaardigheidsdenken deze theorie te funderen: vanuit verlicht eigenbelang zou een burger ook kiezen voor deze regel.
	Zeker met betrekking tot het beschermen van de zwakkeren in de samenleving is deze theorie, volgens de laatste opvatting, goed te verdedigen.

	Het alternatief, absolute aansprakelijkheid, is minder geschikt aangezien niet alle omstandigheden kunnen in acht genomen worden en de consument zelf ook een zekere verantwoordelijkheid heeft.

	\textit{(Hfdst. 5: Ethiek van de Reclame, p.105-108 en p.28-29 voor uitleg over rechtvaardigheidsdenken)}
\end{answer}

\begin{question}
(10)	Wat kan men leren uit de DC10 case met betrekking tot verantwoordelijkheid voor de kwaliteit van producten? 
\end{question}
\begin{answer}[Dieter]
	De verantwoordelijkheid is vaak gedeeld en mag niet zomaar (ongecontroleerd) overgelaten worden aan enkel het bedrijf.
	Als het bedrijf namelijk enkel vanuit een utilitaristisch standpunt denkt (`Waar kunnen we mee wegkomen dat het grootste nut oplevert?') kan kwaliteit, waaronder ook veiligheid valt, in gevaar komen.
	Om kwaliteit de garanderen moet het bedrijf dus kwaliteit (veiligheid)  als een top prioriteit zien.
	Dit moet doordringen in de volledige bedrijfsomgeving zodat er een bereidheid is om alle waarschuwingen en klachten omtrent kwaliteit aan te pakken.
	Externe regulatie kan hierbij helpen zolang die zelf wordt gevrijwaard van belangenconflicten met betrekking tot het bedrijf.
	\textit{(Uit: `BE DC-10 Powerpoint resentation.ppt', slide 29-34)}
\end{answer}

\begin{question}
(11)	Toon aan dat het voorkeurrecht een grote rol kan spelen bij pogingen om een onderneming over te nemen (zie: rechten van aandeelhouders)
\end{question}
\begin{answer}[Dieter]
	Het voorkeursrecht bestaat erin dat bestaand aandeelhouders bij voorkeur kunnen inschrijven op een kapitaalverhoging tegen inbreng van geld in verhouding tot het deel van het kapitaal dat hun aandelen vertegenwoordigt.
	In uitzonderlijke gevallen, zoals bij dreiging tot overname, kan er van dit recht worden afgeweken.
	Op die manier kan men snel reageren op een overnamebod van een grote aandeelhouder door middel van een kapitaalverhoging die het stemmenaantal van die aandeelhouder uitdunt.
	Daardoor kan echter de stem van kleine aandeelhouders dreigen te verwateren en is er tegenwoordig een limiet dat het uitgeven van extra aandelen, in dat geval, tot tien procent beperkt.
	\textit{(Hfdst.7 Rechten en plichten van aandeelhouders, p.170-173)}
\end{answer}

\begin{question}
(12)	Waarom is eigendom niet de meest geschikte invalshoek voor het bepalen van de plichten van de aandeelhouders? Wat is het beste alternatief?
\end{question}
\begin{answer}[Dieter]
	Hoewel het logisch is dat een mede-eigenaar plichten heeft in verband met eigendom, is gewoon eigendom geen goede verdeelsleutel voor verantwoordelijkheid.
	Men kan immers niet zomaar alle aandeelhouders op gelijke wijze verantwoordelijk stellen.
	Vooral grootte aandeelhouders (vaak houdstermaatschapijen) bepalen het beleid en moeten dus, logischerwijs, meer verantwoordelijkheid dragen.
	Een eerste alternatief is het kijken naar de effectieve macht die wordt uitgeoefend.
	Zo zullen managers (van houdstermaatschapijen) vaak effectief invloed kunnen hebben op het beleid.
	Daardoor zullen zij een zwaardere verantwoordelijkheid moeten dragen.
	Een tweede alternatief is de verantwoordelijkheid benaderen vanuit het statuut als burger in de samenleving. Iedereen ongeacht het aantal aandelen heeft namelijk burgerplichten ten opzichte van de medeburgers en de samenleving (o.a. niemand schade toebrengen).
	Een aandeelhouder zou zo via zijn rechten zijn burgerplicht moeten vervullen en de bedrijven waarin wordt geparticipeerd dwingen om ethisch te handelen.
	We achten dit laatste alternatief de beste optie aangezien dit een universeel karakter heeft en ongeacht de macht die de aandeelhouder kan uitoefenen geldt.
\end{answer}

\begin{question}
(13)	Hoe kan men het probleem oplossen van het belangenconflict tussen aandeelhouders en management?  
\end{question}
\begin{answer}[Dieter]
	Aandeelhouders kunnen stemmen tijdens de algemene aandeelhoudersvergaderingen om te voorkomen dat het belangenconflict negatief voor hun uitdraait. 
	Ten tweede kunnen ze zelf alternatieve oplossingen voorstellen.
	\textit{(Rights and Duties of Shareholders.ppt, slides 13)}
\end{answer}

\begin{question}
(14)	Wat is `corporate governance'? Wat is de ethische relevantie van dit begrip?  
\end{question}
\begin{answer}[Dieter]
	\emph{Corporate governance} is een manier om bedrijven, via niet-uitvoerende machten, te sturen en ervoor te zorgen dat er duidelijke communicatie is tussen verschillende stakeholders.
	Zo kunnen aandeelhouders of ethische beleggingsfondsen druk uitoefen op bedrijven die op de een of andere manier onethisch te werk gaan. Een goed voorbeeld hiervan is de druk die werd uitgeoefend op bedrijven die (on)rechtstreeks het Apartheids-regime in Zuid-Afrika steunden.
	\textit{(Rights and Duties of Shareholders.ppt, slides 10, 19)}
\end{answer}

\begin{question}
(15)	Schets de problematiek van het maatschappelijk verantwoord ondernemen.
\end{question}
\begin{answer}[]
\textcolor{red}{TODO}%TODO
\end{answer}

\begin{question}
(16)	Schets de problematiek van het ethisch beleggen.
\end{question}
\begin{answer}[]
\textcolor{red}{TODO}%TODO
\end{answer}

\begin{question}
(17)	De theorie van het sociale contract bij Donaldson. 
\end{question}
\begin{answer}[]
	Het sociaal contract is een toepassing van het rechtvaardigheidsdenken met betrekking tot de ondernemingswereld.
	Het staat is een fictief contract tussen individuele burgers om hun wederzijdse belangen te vrijwaren.
	\textit{(Hfdst. 1: Modellen van bedrijfsethiek en ethische argumenteren, p.22)}
\end{answer}

\begin{question}
(18)	Positieve actie: waarom is dat ethisch noodzakelijk als antwoord op discriminatie.
\end{question}
\begin{answer}[]
\textcolor{red}{TODO}%TODO
\end{answer}

\begin{question}
(19)	Mag een bedrijf een personeelslid aan een H.I.V. test onderwerpen? 
\end{question}
\begin{answer}[Dieter]
	Onder bescherming van de privacy valt ook de bescherming van de medische gegevens van de werknemer.
	De werkgever heeft wel het recht om te informeren naar medische gegevens die rechtstreeks betrekking hebben op de arbeidsgeschiktheid.
	Dat kan echter door een arts gebeuren die enkel melding moet geven van de arbeids(on)geschiktheid.
	In het geval van een werknemer die arts of verplegend personeel is met H.I.V. kan het wel verantwoord zijn om de betrokkene uit het arbeidscircuit te weren.
	\textit{(Hfdst. 6: Human Resources Management, p.158-159)}
\end{answer}

\begin{question}
(20)	Schets de problematiek van de privacy in de onderneming.
\end{question}
\begin{answer}[]
\textcolor{red}{TODO}%TODO
\end{answer}

\begin{question}
(21)	Mag men mensen in een bedrijf met bijzondere technische middelen `bewaken' (camera's, aftappen telefoons, inkijken e-mail en internetverkeer)?
\end{question}
\begin{answer}[Dieter]
	De werkgever mag toezien dat de geleverde arbeid voldoet aan de afgesproken eisen.
	Maar deze mag niet op vergaande wijze inzage en controle uitoefenen op de werknemers en zo hun privacy schenden.
	Standaard zal men dus niet met bijzondere technische middelen mogen bewaken aangezien het recht op privacy bovendien een mensenrecht is dat opgenomen is in de \emph{Universal Declaration of Human Rights}.
	\textit{(Hfdst. 6: Human Resources Management, p. 155)}
\end{answer}

\begin{question}
(22)	Wat kan men leren uit de asbest case met betrekking tot veilige werkomstandigheden? 
\end{question}
\begin{answer}[Dieter]
	Dat het de verantwoordelijkheid is van de werkgever om de risico's te beperken EN de werknemer voldoende te informeren van de risico's.
	Beiden waren niet het geval in de asbest casus aangezien Karel Vinck zich zowel bewust was van de gevaren maar niet voldoende daarover communiceerde en bovendien geen degelijke voorzorgsmaatregelen trof.
	Bovendien werden de werknemers niet voldoende gecompenseerd in hun loon voor het gevaarlijke werk dat ze deden.
\end{answer}

\begin{question}
(23)	Reclame moet waarheidsgetrouw zijn. Leg uit.
\end{question}
\begin{answer}[]
\textcolor{red}{TODO}%TODO
\end{answer}

\begin{question}
(24)	Leg uit: het tekenkarakter van consumptiegoederen. 
\end{question}
\begin{answer}[Dieter]
	Consumptiegoederen worden tegenwoordig vaak aangewend als een teken van status, om zich te onderscheiden van anderen, om trendy te zijn, om zichzelf schijnbaar bij een groep te doen horen die hoger staat aangeschreven.
	Vanaf het goed wijd genoeg verspreid is en nauwelijks nog een verschil maakt zal er volgens deze logica een nieuw consumptiegoed moeten aangekocht worden om weer onderscheid te kunnen maken (bvb. tv differentieert niet meer).
	Dit leid tot een cultuur waarin de identiteit van mensen verarmd wordt tot hun vermogen om consuptiegoederen te kopen.
	Er wordt op die manier bovendien een subjectieve schaarste van goederen gecre\"eerd waardoor er weinig aandacht en middelen overblijven voor de mensen wiens basisbehoeften nog niet bevredigd zijn. 
	\textit{(Hfdst. 5: Ethiek van reclame, p.119-120 en BEAdvertising.ppt, slide 13)}
\end{answer}

\begin{question}
(25)	Wat hebben `merken' (brands) met religie te maken? 
\end{question}
\begin{answer}[Dieter]
	Niet religie maar adverteren is tegenwoordig het `opium van het volk'. Namelijk een belofte van een denkbeeldige wereld van pracht en geluk. Advertenties beloven hoop. Verder wordt religie soms ook gebanaliseerd door advertenties bijvoorbeeld door het gebruiken van beelden als het laatste avondmaal. \textit{(BEAdvertising.ppt, slide 18)}
\end{answer}

\begin{question}
(26)	Wat betekent het begrip stakeholder. De sterke en zwakke punten van het stakeholder-model.
\end{question}
\begin{answer}[]
\textcolor{red}{TODO}%TODO
\end{answer}

\begin{question}
(27)	Wat kan je leren van de casus van het ruimteveer Challenger met betrekking tot \'e\'en van de meest fundamentele spanningen in het leven van een ingenieur (ingenieur versus manager)? 
\end{question}
\begin{answer}[]
\textcolor{red}{TODO}%TODO
\end{answer}

\begin{question}
(28)	Waarom is de bedrijfsrevisor (boekhoudkundige controle) zo belangrijk voor de aandeelhouders en de raad van bestuur van een onderneming? Verhelder je antwoord met een voorbeeld.
\end{question}
\begin{answer}[]
\textcolor{red}{TODO}%TODO
\end{answer}

\begin{question}
(29)	Waarom is het voorzorgprincipe belangrijker dan het principe `de vervuiler betaalt'? 
\end{question}
\begin{answer}[]
\textcolor{red}{TODO}%TODO
\end{answer}

\begin{question}
(30)	Leg uit hoe Hans Jonas (Das Prinzip Verantwortung) de zorg voor de toekomstige generaties ethisch onderbouwt.
\end{question}
\begin{answer}[Dieter]
	Hans Jonas gebied dat men moet handelen zodat de effecten van de uitgevoerde acties compatibel zijn met het duurzaamheid van echt menselijk leven op de aarde.
	Dit is zeer gelijkaardig aan het voorzorgprincipe aangezien als een actie zelfs een geringe kans heeft om de duurzaamheid van echt menselijk leven te schaden die beter niet wordt uitgevoerd.
	\textit{(Nano-ethics.ppt, slide 27)}
\end{answer}

\begin{question}
(31)	Wat heeft leiderschap met spiritualiteit te maken? 
\end{question}
\begin{answer}[]
\textcolor{red}{TODO}%TODO
\end{answer}

\begin{question}
(32)	Wat heeft het gebrek aan inspirerend leiderschap met taal te maken? 
\end{question}
\begin{answer}[Dieter]
	De mens is een taaldier.
	Via taal interpreteren en manipuleren we de werkelijkheid. 
	Het verschralen van de taal vermindert de betekenis van wat wij doen.
	Daardoor is louter spreken in vaktaal gevaarlijk (tunnelvisie).
	Daarom is er voor goed leiderschap een tweede taal nodig: taal van het verstaan, metafoor, verhaal.
	Deze drukt uit wat mensen inspireert, bezielt, esthetische en liefdes ervaringen, \dots
	Deze tweede taal zal kunnen inspiriren en dat is wat goede leiders doen.
	\textit{(Uit lederschapcursusbe.ppt)}
\end{answer}

\begin{question}
(33)	Hoe kan spiritualiteit bijdragen tot minder afhankelijkheidsgedrag? 
\end{question}
\begin{answer}[]
\textcolor{red}{TODO}%TODO
\end{answer}

\nocite{*}
\printbibliography

\end{document}
